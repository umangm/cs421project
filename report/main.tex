\documentclass{article}

\usepackage[top=1in, left=1in, right=1in]{geometry}
\usepackage[utf8]{inputenc}
\usepackage{courier}
\usepackage{hyperref}
\usepackage{listings}
\usepackage{color}

%New colors defined below
\definecolor{codegreen}{rgb}{0,0.6,0}
\definecolor{codegray}{rgb}{0.5,0.5,0.5}
\definecolor{codepurple}{rgb}{0.58,0,0.82}
\definecolor{backcolour}{rgb}{0.95,0.95,0.92}

\makeatletter
\lst@InstallKeywords k{types}{typestyle}\slshape{typestyle}{}ld
\makeatletter

%Code listing style named "mystyle"
\lstdefinestyle{mystyle}{
  backgroundcolor=\color{backcolour},   commentstyle=\color{blue},
  keywordstyle=\color{red},
  numberstyle=\tiny\color{codepurple},
  stringstyle=\color{codepurple},
  basicstyle=\footnotesize\ttfamily,
  breakatwhitespace=false,         
  breaklines=true,                 
  captionpos=b,                    
  keepspaces=true,                 
  numbers=left,                    
  numbersep=5pt,                  
  showspaces=false,                
  showstringspaces=false,
  showtabs=false,                  
  tabsize=2,
}

%"mystyle" code listing set
\lstset{style=mystyle}

\lstset{emph={
  int, float, string, bool, list
    , List, String, Char
    , Failure
    , TrueConst
    , FalseConst
    , IntConst
    , FloatConst
    , StringConst
    , NilConst
    , UnitConst
    , IntPlusOp , IntMinusOp , IntTimesOp , IntDivOp , FloatPlusOp , FloatMinusOp , FloatTimesOp , FloatDivOp , ConcatOp , ConsOp , CommaOp , EqOp , GreaterOp , ModOp , ExpoOp
    , HdOp , TlOp , PrintOp , IntNegOp , FstOp , SndOp
    , VarExp   , ConstExp   , MonOpAppExp   , BinOpAppExp    , IfExp , AppExp , FunExp , LetInExp , LetRecInExp    , RaiseExp  , TryWithExp
    , Anon, Let, LetRec
    , None, Some
    , TyVar, TyConst
    , ExpJudgment , DecJudgment
    , Proof
    , UnitVal , TrueVal , FalseVal , IntVal , FloatVal, StringVal, PairVal, Closure, ListVal , RecVarVal , Exn
    , ContVarCPS, External, FnContCPS, ExnMatch
    , ExnContVarCPS, EmptyExnContCPS, UpdateExnContCPS
    , VarCPS, ConstCPS, MonOpAppCPS, BinOpAppCPS, IfCPS, AppCPS, FunCPS, FixCPS
    }, emphstyle={\color{codegreen}}
}



\title{Checking Tail Recursion in PicoML}
\author{Umang Mathur \\ \href{mailto:umathur3@illinois.edu}{umathur3@illinois.edu}
\and Chia-Hao Hsieh \\ \href{mailto:chsieh17@illinois.edu}{chsieh17@illinois.edu}}
\date{\today}

\begin{document}

\maketitle

\section{Overview}

\subsection{Recursion}
\hspace{0.2in} 
The use of recursion dates back to the late 19$^{th}$ century, when mathematicians Dedekind and Peano used induction to defined functions.
The use of recursion played an important role in foundations of computer science, and was later referred to as 'primitive recursion'~\cite{Soare96}

Use of recursion is not just exciting from the perspective of a Mathematician, but is also quite significant from the perspective of a developer.
Allowing procedures to be recursive helps the programmer write more readable and intuitive/natural programs.
A notable use of recursion is seen when dealing with inductive structures.
Inductive definitions and inductive programs can be very naturally programmed as recursive functions.
Besides, recursive functions, can be easier to debug, due to the same reason.
Recursive programs, at times, tend to be more efficient than a naive program with loops and no recursive calls. 
Recursion, thus is a very handy tool for programmers.

\subsection{Checking Tail Reclusion : Motivation}

\hspace{0.2in} The convenience offered due to recursion, comes at a cost. 
Recursive programs are generally modelled by the use of stack frames. 
This means that recursive programs tend to consume extra space (stack) for every recursive call they make.
Besides, the additional overhead of copying the variables and values to the new frame, also accounts for a non trivial overhead, which at times, is not desirable from the standpoint of efficiency.

However, the extra space consumed can be overcome when the recursive call is the last thing the function does. 
In this case, the contents of the stack can be replaced by the new frame, and there is no need to push an additional frame.

The idea behind tail call optimization is essentially the same. 
Informally, a recursive function is tail recursive when the recursive call is the last thing executed by the function. 
Thus, if the compiler can detect if a function is tail-recursive, it can convert the function to an equivalent while-loop, thus avoiding an additional call that consumes extra stack space by virtue of the new frame added.

\subsection{Goal of the Project}
In this project, we implement a tool \tool that checks if a procedure is tail recursive or not.
Specifically, we wish to analyze declarations written in PicoML. 
PicoML is a restricted form of OCaml, and supports simple expressions like  $\ifexp{\!}{\!}{\!}$, $\mathtt{fun}$, $\mathtt{let} \! \mathtt{rec}$. 
As part of the assignments in the course, we have built an interpreter for this language~\cite{OlderMP}.
We aim to integrate the \tool with the interpreter. 
That is, we would use the parsing and the type checking functionality written in older assignments.
This would enable use to directly use the functionality for implementing \tool, and would save some  effort, 
as compared to the scenario where we had to re-invent the wheel.

\newpage

\section{Definitions}
Before we describe our implementation, it would be useful to go through some notations and definitions.\\
The purpose of the definitions is to give a nice characterization of the problem we wish to address.\\

\begin{definition}[PicoML Expression]
A PicoML-style expression $e$ is formally defined by the recursive grammar:
\begin{equation*}
\begin{split}
e :=& \; c \; | \; v \; | \; \odot e \; | \; e \oplus e \; | \; \ifexp{e}{e}{e} \; | \; e \; e \; | \; \funexp{x}{e} \; \\
  &| \; \letinexp{x}{e}{e} \; | \; \letrecinexp{f}{x}{e}{e}  \; | \; \raiseexp{e} \;\\
  &| \; \mathtt{try} \; e \; \mathtt{with} \; \mathtt{e\_int\_list}
\end{split}
\end{equation*}

where, $c$ is a constant, $v$ is a variable, and $\mathtt{e\_int\_list}$ is inductively defined as:

\begin{equation*}
\begin{split}
\mathtt{e\_int\_list} :=& \; \mathtt{i} \arr e  \; | \; \mathtt{i} \arr e , \mathtt{e\_int\_list}
\end{split}
\end{equation*}

where $i$ is an integer
\end{definition}

\begin{definition}[PicoML Declaration]
Declarations in PicoML are declarative statements that assign an expression to an identifier. Alternately, then can also be plain expressions.

\begin{equation*}
\begin{split}
\mathtt{dec} :=& \; e  \; | \; \mathtt{let} \; x \; \mathtt{=} \; e \; | \; \mathtt{let} \; \mathtt{rec} \; f \; x \; \mathtt{=} \; e
\end{split}
\end{equation*}
\end{definition}


\begin{definition}[Recursive Expression]
An expression $e$ defined in PicoML is defined to be recursive with respect to an identifier $f$ if 
there is a subexpression $e'$ of $e$ (that is not lambda lifted) and has the form $f \; e''$, where $e''$ is an expression, 
and no expression that contains $e'$ redefines $f$
\end{definition}

Let us take a look at a couple of examples to understand the above definition.
\begin{example}
Consider the PicoML declaration below:
\[
\mathtt{let} \; \mathtt{rec} \; f \; x \; = \; \ifexp{x = 0}{1}{f \; (f \; x - 1)} ;;
\]
Note that, the body of the above declaration is recursive in $f$, based on the above definition.
 \end{example} 

\begin{example}
\label{ex:caveat}
Consider the expression defined below:
\[
\mathtt{let} \; {g} \; = f \; \mathtt{in} \; g \; 3;; 
\]
Note that, based on the definition above, the above expression is not recursive in $f$
 \end{example} 

\begin{definition}{Tail Recursive Expression}
\label{def:exptailrec}
A PicoML expression $e$ is said to be tail-recursive with respect to an identifier $f$ if it is recursive in $f$ and one of the following holds:
\begin{itemize}
\item $e$ is a constant $c$ or a variable $v$
\item $e$ is of the form $\odot e'$, and $e'$ is not recursive in $f$
\item $e$ is of the form $e' \oplus e''$, and none of $e'$ and $e''$ is recursive in $f$
\item $e$ is of the form $e' e''$ and $e''$ is not recursive in $f$ and $e'$ is tail-recursive in $f$
\item $e$ is of the form $\ifexp{e'}{e''}{e'''}$ if $e'$ is not recursive in $f$, and both $e''$ and $e'''$ are tail recursive in $f$
\item $e$ is of the form $\funexp{x}{e'}$
\item $e$ is of the form $\letinexp{x}{e'}{e''}$ and, either
  \begin{itemize}
   \item $x$ is not $f$ (that is, f has the older binding in $e''$), $e'$ is not recursive in $f$, and $e''$ is tail recursive in $f$, or
   \item $x$ is $f$ (that is, the binding of $f$ is updated in $e''$), and $e'$ is tail recursive in $f$
  \end{itemize}
\item $e$ is of the form $\letrecinexp{g}{x}{e'}{e''}$, and, one of the following is true
  \begin{itemize}
   \item $g$ is $f$ (that is, f is given a new binding in $e''$), or
   \item $g$ is not $f$, (here, f retains its binding in $e''$), and $e''$ is tail recursive in $f$
  \end{itemize}
\item $e$ is of the form $\mathtt{try} \; e' \; \mathtt{with} \; i1 \arr e_{i1} \; | \; i2 \arr e_{i2} \; | \; \ldots \; | \; ik \arr e_{ik}$, and
  \begin{itemize}
   \item $e'$ is not tail recursive in f, and
   \item Each of $e_{i1}$ through $e_{ik}$ is tail recursive in $f$
  \end{itemize}
\end{itemize}
\end{definition}

Note that, as discussed in Example~\ref{ex:caveat}, the above characterization does not check for tail recursion
if the expression is tail recursive by virtue of aliases. While this takes our characterization further from being a complete characterization, 
we do not compromise the soundness. That is to say, that if an expression is reported to tail recursive using our characterization, it cannot be a false alarm.


\begin{definition}[Tail Recursive Declaration]
\label{def:dectailrec}
A PicoML declaration is said to be tail recursive if one of the following hold:
\begin{itemize}
 \item It is of the form $e$
 \item It is of the form $\mathtt{let} \; x \; \mathtt{=} \; e$
 \item It is of the form $\mathtt{let} \; \mathtt{rec} \; f \; x \; \mathtt{=} \; e$ and $e$ is tail recursive in $f$
\end{itemize}
\end{definition}


\newpage 

\section{Implementation}
For the purpose of the project, we check if a PicoML declaration is  tail recursive or not, in the following two ways :

\begin{enumerate}
 \item We check if the declaration satisfies Definition~\ref{def:dectailrec} given earlier
 \item We transform the declaration into Continuation Passing style, and check for some conditions (covered later)
\end{enumerate}

\subsection{Checking Tail Recursion in PicoML}

For the, purpose of checking tail recursion in \tool
we need to ensure that whenever an expression satisfies the conditions in Definition~\ref{def:exptailrec}, we should report so,
and report 'not tail recursive' otherwise.
In order to ensure that this is the case, our implementation has to be faithful to the definition.
Because of the inductive nature of the definition, it becomes lot easier to implement the algorithm.

Here is a brief snapshot of the algorithm:

\begin{lstlisting}[language=Caml, caption=Tail recursion for PicoML expressions]
let rec check_tail_rec_f f e =
    match e
    with ConstExp c -> true
    | VarExp v -> true
    | MonOpAppExp (mon_op, e1) -> not (check_rec_f f e1) 
    | BinOpAppExp (bin_op, e1, e2) -> (not (check_rec_f f e1)) && (not (check_rec_f f e2))
    | AppExp(e1, e2) -> if (check_rec_f f e2) then false else check_tail_rec_f f e1
    | IfExp (e1, e2, e3) -> (not (check_rec_f f e1)) && (check_tail_rec_f f e2) && (check_tail_rec_f f e3)
    | FunExp (x, e1) -> true 
    | LetInExp (x, e1, e2) -> if (x = f) then (not (check_rec_f f e1)) else ( (not (check_rec_f f e1)) && (check_tail_rec_f f e2))
    | LetRecInExp (g, x, e1, e2) ->
        if (g = f) then true 
        else if (not (g=f) && (x=f)) then (check_tail_rec_f f e2)
        else ( (check_tail_rec_f f e2)) 
    | TryWithExp (e', n1opt, e1, nopt_e_lst) ->
        if (check_rec_f f e') then false
            else let lst = ((n1opt, e1)::nopt_e_lst) in (List.fold_right (fun (intop, h) -> fun t -> (check_tail_rec_f f h) && t) lst true) 
    | _ -> false ;;
\end{lstlisting} 

\subsection{Checking Tail Recursion in CPS expressions}

Checking if a CPS transformed expression is tail recursive in $f$ basically amounts to checking if
an application of $f$ is passed onto its original continuation (which happens to be a trivial check for our framework, because the
original continuation is an artificially added dummy continuation, having a special signature).

The idea behind this is that if the continuation for an application of $f$ is another function, it basically means that
the application is not in the tail position, and there is some \emph{work to be done} after the application.
\\\\
Below is a snippet of the relevant part of the code:

\begin{lstlisting}[language=Caml, caption=Tail recursion for CPS transformed expressions]
let rec check_cps_tail_rec_f flist original_k x k e = 
    match e
    with ConstCPS (k', c) -> cont_tail_recursive original_k k' flist
    | VarCPS (k', v) -> cont_tail_recursive original_k k' flist
    | MonOpAppCPS (k', mono_op, o1, exk) -> cont_tail_recursive original_k k' flist
    | BinOpAppCPS (k', bin_op, o1, o2, exk) -> cont_tail_recursive original_k k' flist
    | IfCPS (b, e1, e2) -> (check_cps_tail_rec_f flist original_k x k e1) && (check_cps_tail_rec_f flist original_k x k e2)
    | AppCPS (k', e1, e2, exk) -> 
        if ( List.exists (fun x -> x = e1) flist)
        then (k'=original_k)
        else cont_tail_recursive original_k k' flist
    | FunCPS (kappa, x, k, ek, e) -> true
    | FixCPS (kappa, f, x, k, ek, e) -> true
and cont_tail_recursive original_k k flist = 
    match k
    with ContVarCPS i -> true
    | External -> true
    | FnContCPS (x, e) -> 
        check_cps_tail_rec_f flist original_k x k e
    | ExnMatch ek -> true ;;
\end{lstlisting}


\subsection{Code Structure}

The tool \tool has been developed in a series of iterations, and deliberate attempts have been made to keep
the code concise and short.
The total lines of code is about 1800, and the central parts of the code (where we implement the main algorithms), is about 200 lines (100 lines each for direct style and CPS style)

The code has been kept modular, to the best possible extent.
The following is a brief description of the various files:
\begin{itemize}
 \item \texttt{definitions.ml} ; Includes the various definitions for the types used in the tool. 
 This file includes algorithms for unification, type-inference, transforming PicoML expressions in CPS etc., apart from basic utilities for printing various types.
 Most of the parts in this file have been derived from MP7
 \item \texttt{tailRecPicoMLlex.mll} : Includes the code for token generation for PicoML expressions, largely borrowed from ML4
 \item \texttt{tailRecPicoMLparse.mly} : Includes the code for parsing PicoML expressions, largely borrowed from MP5
 \item \texttt{checkTailRec.ml} : Contains the main algorithm for checking PicoML expressions and declarations for tail recursion
 \item \texttt{checkTailRecCPS.ml} : Contains the main algorithm for checking CPS expressions and declarations for tail recursion
 \item \texttt{tailRecPicoMLInt.ml} : Contains utilities for wrapping the main code for interactive use
 \item \texttt{tailRecPicoMLTest.ml} : Contains utilities for wrapping the main code for non-interactive testing
\end{itemize}

\subsection{Comparison with Original Proposal}

Below is the relevant part of the original proposal :

\lstinputlisting[language={}]{proposal.txt}

Clearly, we have improved upon the proposal in the following ways: 

\begin{enumerate}
 \item We have changed a lot of definitions, realizing that the earlier characterizations were unsound, or far from complete, or completely wrong.
 Specific differences have not been pointed out for the sake of conciseness
 \item We have also implemented an algorithm for checking CPS expressions, thus enabling us to check for robustness of the tool.
\end{enumerate}


\newpage

\section{Tests}

Each line in the file \texttt{testing.txt} is a test case. 
To see testing results, run \texttt{./tailRecPicoMLTest} after doing a  \texttt{make}

Test cases can be categorized as follows:

\subsection{Smoke Tests}

\begin{lstlisting}[language=Caml, caption=Smoke Tests]
5;; (* for dec = (Anon e) *)
let f = 5;; (* for dec = Let (s, e) *)
let rec f x = 100;; (* smoke test for ConstExp *)
let rec f x = x;; (* smoke test for VarExp *)
let rec f x = hd x;; (* smoke test for MonOp *)
let rec f x = x + x;; (* smoke test for BinOp *)
let rec f x = f 5;; (* smoke test for AppExp *)
let rec f x = (fun x -> x);; (* smoke test for FunExp *)
\end{lstlisting}

\subsection{Typical Example Cases}

\begin{lstlisting}[language=Caml, caption=Typical Example Cases]
let rec f x = if (x=0) then 0 else f (x-1);; (* typical tail-recursive example *)
let rec f x = if (x=0) then 0 else x * (f (x-1));; (* typical not tail-recursive example *)
\end{lstlisting}

\subsection{More Test Cases}

\begin{lstlisting}[language=Caml, caption=More Test Cases]
let rec f x = if (x=0) then 0 else f(f(f(f (x-1))));; (* a variavation of typical tail-recursive example *)
let rec f x = let f = 5 in f + f;; (* for LetIn: redefine f *)
let rec f x = if (x=0) then 0+0+0+0 else (f (x-1));; (* for BinOp *)
let rec f x = (fun y -> 0) x;; (* more cases for FunExp and AppExp *)
let rec f x = (fun x -> x * x) x ;;
let rec f x = (fun x -> x * (f x)) x ;;
let rec f x = let x_0=(x=0) in if x_0 then 0 else (f (x-1));; (* more cases for LetIn *)
let rec f x = let f_x_1=(f (x-1)) in if x = 0 then 0 else f_x_1;;
let rec f x = let g = (f (x-1)) in if (x=0) then 0 else g;;
let rec f x = let g = x-1 in if (x=0) then 0 else f g;;
let rec f x = let g = (f x) + (f x) in if (x=0) then 0 else (f (x-1));;
let rec f x = let g = (f x) + (f x) in if (x=0) then 0 else g;;
let rec f x = let rec g y = y * y in if (x=0) then (g 0) else (f (x-1));; (* more cases for LetRecIn *)
let rec f x = let rec f y = 0 in if (x=0) then (f 0) else (f (x-1));;
let rec f x = let rec g f = 0 in if (x=0) then (g 0) else (f (x-1));;
let rec f x = let rec g y = (f y) in if (x=0) then 0 else (f (x-1));;
let rec f x = let rec g y = (f y) + (f y) in if (x=0) then 0 else g x;;
let rec f x = let rec g y = (f y) in if (x=0) then 0 else (g (x-1));;
\end{lstlisting}

\subsection{Test Cases that aren't handled well}

The following test cases, are known to be either failing, or pass without the correct reason (that is, they do not fit out characterization for the correct reason)
\begin{lstlisting}[language=Caml, caption=Unhandled Test Cases]
let rec f x = let f = f in if (x=0) then 0 else (f (x-1));; (* more cases for LetIn: aliases *)
let rec f x = let g = f in if (x=0) then 0 else g (x-1);;
let rec f x = let g = f in g x;;
let rec f x = let g = f in if (x=0) then 0 else (g (x-1));;
let rec f x = let g = f in if (x=0) then 0 else (x-1)*(g (x-1));; (* Failed with both styles since we can't handle aliases well *)
let rec f x = let g = f in let h = g in if (x=0) then 0 else (h (x-1))*(x-1);; (* Failed with both styles since we can't handle aliases well *)
\end{lstlisting}


\newpage

\section{Listing}

This sections gives a listing of the code written for the tool \tool

\subsection{checkTailRec.ml}

\input{codelisting/checkTailRec.ml}

\newpage

\subsection{checkTailRecCPS.ml}

(*
checkTailRecCPS.ml - DO NOT EDIT
 *)

open Definitions

let rec convert_f_exp e name_of_f = 
    (* Return all variable numbers that correspond to f. *)
    (
    match e
    with  ConstCPS (k, c) -> 
        convert_f_cont k name_of_f
    | VarCPS (k, g) -> 
        (*
        print_string "\nVarCPS \n";
        *)
            (
            match k 
            with FnContCPS (number_of_f, e') -> 
                let rec_list = convert_f_exp e' name_of_f 
                in 
                if (g = name_of_f) 
                    (* Found a matched f. Add the number to retured list. *)
                    then (number_of_f :: rec_list) 
                    else rec_list
            | _ -> []
            )
    | MonOpAppCPS (k, _, _, _) -> 
        (*
        print_string "\nMonOpAppCPS \n";
        *)
        convert_f_cont k name_of_f
    | BinOpAppCPS (k , _, _, _, _) -> 
        (*
        print_string "\nBinOpAppCPS \n";
        *)
        convert_f_cont k name_of_f
    | IfCPS (b, e1, e2) ->
        (*
        print_string "\nIfCPS \n";
        *)
        (convert_f_exp e1 name_of_f)@(convert_f_exp e2 name_of_f)
    | AppCPS (k, _, _, _) ->
        (*
        print_string "\nAppCPS \n";
        *)
        convert_f_cont k name_of_f
    | FunCPS (k, _, _, _, _) ->
        (*
        print_string "\nFunCPS \n";
        *)
        convert_f_cont k name_of_f
    | FixCPS (k, _, _, _, _, _) ->
        (*
        print_string "\nFixCPS \n";
        *)
        convert_f_cont k name_of_f
    )

and 

convert_f_cont k name_of_f =
    (
    match k
    with FnContCPS (_, e') -> 
        convert_f_exp e' name_of_f
    | _ -> []
    ) ;;


let rec check_cps_tail_rec_f flist original_k x k e = 
    match e
    with ConstCPS (k', c) -> 
        (* It's impossilbe to call f here, so go deeper to find if there are any f's. *)
        cont_tail_recursive original_k k' flist
    | VarCPS (k', v) -> 
        (* Similarly, go deeper to find f. *)
        cont_tail_recursive original_k k' flist
    | MonOpAppCPS (k', mono_op, o1, exk) ->
        (* Similarly, go deeper to find f. *)
        cont_tail_recursive original_k k' flist
    | BinOpAppCPS (k', bin_op, o1, o2, exk) -> 
        (* Similarly, go deeper to find f. *)
        cont_tail_recursive original_k k' flist
    | IfCPS (b, e1, e2) -> 
        (* Similarly, go deeper to find f. *)
        (check_cps_tail_rec_f flist original_k x k e1) && (check_cps_tail_rec_f flist original_k x k e2)
    | AppCPS (k', e1, e2, exk) -> 
        (
        (* if e1 matches one of variable numbers corresponding to f ... *)
        if ( List.exists (fun x -> x = e1) flist)
            then 
                (
                (* And if the returned k is the original ...*)
                if (k'=original_k) 
                    then true (* then this is a tail call. *)
                    else false
                )
            else
                (* Else just go deeper to find if there are any f's. *)
                cont_tail_recursive original_k k' flist
        )
    | FunCPS (kappa, x, k, ek, e) -> true
    | FixCPS (kappa, f, x, k, ek, e) -> true

and cont_tail_recursive original_k k flist = 
    match k
    with ContVarCPS i -> true
    | External -> true
    | FnContCPS (x, e) -> 
        check_cps_tail_rec_f flist original_k x k e
    | ExnMatch ek -> true ;;


let check_tail_recursion dec =
    match dec
    with Anon e -> true
    | Let (x,e) -> true
    | LetRec (f,x,e) ->
        let (i,j) = (next_index(),next_index()) 
        in
            let ecps2 = cps_exp e (ContVarCPS i) (ExnContVarCPS j) 
            in

            let flist = convert_f_exp ecps2 f
                in
                check_cps_tail_rec_f flist (ContVarCPS i) x (ContVarCPS i) ecps2 ;;


\newpage

\subsection{definitions.ml}

\input{codelisting/definitions.ml}

\newpage

\subsection{tailRecPicoMLInt.ml}

\input{codelisting/tailRecPicoMLInt.ml}

\newpage

\subsection{tailRecPicoMLTest.ml}

\input{codelisting/tailRecPicoMLTest.ml}

\newpage

\subsection{tailRecPicoMLparse.mly}

\input{codelisting/tailRecPicoMLparse.mly}

\newpage

\subsection{tailRecPicoMLlex.ml}

\input{codelisting/tailRecPicoMLlex.mll}

\begin{thebibliography}{9}
  
\bibitem{Soare96}
    Robert I. Soare,
    \emph{Computability and recursion},
    BULL. SYMBOLIC LOGIC,
    1996

\bibitem{OlderMP}
  Programming Languages and Compilers : CS421,
  Fall 2015,
  University of Illinois, Urbana Champaign,
\href{https://courses.engr.illinois.edu/cs421/mps/index.html}{Course Web Page}
\end{thebibliography}


\end{document}
