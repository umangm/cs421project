\begin{enumerate}
 \item We check if the declaration satisfies Definition~\ref{def:dectailrec} given earlier
 \item We transform the declaration into Continuation Passing style, and check for some conditions (covered later)
\end{enumerate}

\subsection{Checking Tail Recursion in PicoML}

For the, purpose of checking tail recursion in \tool
we need to ensure that whenever an expression satisfies the conditions in Definition~\ref{def:exptailrec}, we should report so,
and report 'not tail recursive' otherwise.
In order to ensure that this is the case, our implementation has to be faithful to the definition.
Because of the inductive nature of the definition, it becomes lot easier to implement the algorithm.

Here is a brief snapshot of the algorithm:

\begin{lstlisting}[language=Caml, caption=Tail recursion for PicoML expressions]
open Definitions;;

let rec check_let_in_meaningful x e =
    match e
    with ConstExp c -> false
    | VarExp v -> if (v = x) then true else false
    | MonOpAppExp (mon_op, e1) -> check_let_in_meaningful x e1
    | BinOpAppExp (bin_op, e1, e2) -> (check_let_in_meaningful x e1) || (check_let_in_meaningful x e2) 
    | IfExp (e1, e2, e3) ->
        (check_let_in_meaningful x e1) || (check_let_in_meaningful x e2)  || (check_let_in_meaningful x e3) 
    | LetInExp (s, e1, e2) ->
        if (check_let_in_meaningful x e1)
            then (check_let_in_meaningful s e2)
            else (
                if (x=s) 
                    then false
                    else check_let_in_meaningful x e2
                )
    | FunExp (s, e1) -> if (s=x) then false else (check_let_in_meaningful x e1)
    | AppExp (e1, e2) -> 
        (check_let_in_meaningful x e1) || (check_let_in_meaningful x e2)
    | LetRecInExp (g, y, e1, e2) ->
        if ((g=x) || (y=x)) 
            then false 
            else if (check_let_in_meaningful x e1) 
                then (check_rec_f g e2)
                else (check_let_in_meaningful x e2)
    | RaiseExp e1 -> (check_let_in_meaningful x e1)
    | TryWithExp (e0, n1opt, e1, nopt_e_lst) ->
        (check_let_in_meaningful x e0) || (check_let_in_meaningful_lst x ((n1opt,e1)::nopt_e_lst) )
and check_let_in_meaningful_lst x nopt_e_lst = 
    match nopt_e_lst 
    with [] -> false
    | (nopt, en)::rest -> (check_let_in_meaningful x en) || (check_let_in_meaningful_lst x rest)
and check_rec_f f e =
    match e 
    with ConstExp c -> false
    | VarExp v -> false
    | MonOpAppExp (mon_op, e1) -> check_rec_f f e1
    | BinOpAppExp (bin_op, e1, e2) -> (check_rec_f f e1) || (check_rec_f f e2)
    | IfExp (e1, e2, e3) ->
        (check_rec_f f e1) || (check_rec_f f e2) || (check_rec_f f e3)
    | LetInExp (s, e1, e2) ->
        if (check_rec_f f e1) 
            then (check_let_in_meaningful s e2)
            else
                (
                if (s=f) then false else ( (check_rec_f f e1) || (check_rec_f f e2) )
                )
    | FunExp (s, e1) -> if (s=f) then false else (check_rec_f f e1)
    | AppExp (e1, e2) -> 
        (
        match e1
        with VarExp v -> if (v=f) then true else false
        | _ -> (check_rec_f f e1) || (check_rec_f f e2)
        )
    | LetRecInExp (g, x, e1, e2) -> 
        if ( (g=f) || (x=f)) 
            then false 
            else if (check_rec_f f e1) 
                then (check_rec_f g e2)
                else (check_rec_f f e2)
    | RaiseExp e1 -> (check_rec_f f e1)
    | TryWithExp (e0, n1opt, e1, nopt_e_lst) ->
        (check_rec_f f e0) || (check_rec_f_lst f ((n1opt, e1) :: nopt_e_lst) )

and check_rec_f_lst f nopt_e_lst = 
    match nopt_e_lst 
    with [] -> true
    | ((nnopt, en)::rest) -> ((check_rec_f f en) || (check_rec_f_lst f rest));;

let rec check_tail_rec_f f e =
    match e
    with ConstExp c -> true
    | VarExp v -> true
    | MonOpAppExp (mon_op, e1) -> not (check_rec_f f e1)
    | BinOpAppExp (bin_op, e1, e2) -> (not (check_rec_f f e1)) && (not (check_rec_f f e2))
    | AppExp(e1, e2) -> if (check_rec_f f e2)
        then false
        else check_tail_rec_f f e1
    | IfExp (e1, e2, e3) -> 
            (not (check_rec_f f e1)) &&
            (check_tail_rec_f f e2) && 
            (check_tail_rec_f f e3)
    | FunExp (x, e1) -> true
    | LetInExp (x, e1, e2) ->
        if (x = f) 
            then (not (check_rec_f f e1))
            else ( (not (check_rec_f f e1)) && (check_tail_rec_f f e2))
    | LetRecInExp (g, x, e1, e2) ->
        if (g = f) then true
        else if (not (g=f) && (x=f)) then (check_tail_rec_f f e2)
        else ( (check_tail_rec_f f e2))
    (* Before fix:
        if (g = f) then true
        else if (not (g=f) && (x=f)) then (check_tail_rec_f f e2)
        else ( (not (check_rec_f f e1)) && (check_tail_rec_f f e2))
    *)
    | TryWithExp (e', n1opt, e1, nopt_e_lst) ->
        (
        if (check_rec_f f e')
            then false
            else let lst = ((n1opt, e1)::nopt_e_lst)
                in
                (List.fold_right (fun (intop, h) -> fun t -> (check_tail_rec_f f h) && t) lst true) 
        )
    | _ -> false ;;


let check_tail_recursion dec =
    match dec
    with (Anon e) -> true
    | Let (s, e) -> true
    | LetRec (f, x, e) ->
        check_tail_rec_f f e ;;
\end{lstlisting} 

\subsection{Checking Tail Recursion in CPS expressions}

Checking if a CPS transformed expression is tail recursive in $f$ basically amounts to checking if
an application of $f$ is passed onto its original continuation (which happens to be a trivial check for our framework, because the
original continuation is an artificially added dummy continuation, having a special signature).

The idea behind this is that if the continuation for an application of $f$ is another function, it basically means that
the application is not in the tail position, and there is some \emph{work to be done} after the application.
\\\\
Below is a snippet of the relevant part of the code:

\begin{lstlisting}[language=Caml, caption=Tail recursion for CPS transformed expressions]

open Definitions

let rec convert_f_exp e name_of_f = 
    (
    match e
    with  ConstCPS (k, c) -> 
        (*
         print_string "\nConstCPS \n";
        *)
        convert_f_cont k name_of_f
    | VarCPS (k, g) -> 
        (*
        print_string "\nVarCPS \n";
        *)
            (
            match k 
            with FnContCPS (number_of_f, e') -> 
                let rec_list = convert_f_exp e' name_of_f 
                in 
                if (g = name_of_f) 
                        then (number_of_f :: rec_list) 
                        else rec_list
            | _ -> []
            )
    | MonOpAppCPS (k, _, _, _) -> 
        (*
        print_string "\nMonOpAppCPS \n";
        *)
        convert_f_cont k name_of_f
    | BinOpAppCPS (k , _, _, _, _) -> 
        (*
        print_string "\nBinOpAppCPS \n";
        *)
        convert_f_cont k name_of_f
    | IfCPS (b, e1, e2) ->
        (*
        print_string "\nIfCPS \n";
        *)
        (convert_f_exp e1 name_of_f)@(convert_f_exp e2 name_of_f)
    | AppCPS (k, _, _, _) ->
        (*
        print_string "\nAppCPS \n";
        *)
        convert_f_cont k name_of_f
    | FunCPS (k, _, _, _, _) ->
        (*
        print_string "\nFunCPS \n";
        *)
        convert_f_cont k name_of_f
    | FixCPS (k, _, _, _, _, _) ->
        (*
        print_string "\nFixCPS \n";
        *)
        convert_f_cont k name_of_f
    )

and 

convert_f_cont k name_of_f =
    (
    match k
    with FnContCPS (_, e') -> 
        convert_f_exp e' name_of_f
    | _ -> []
    ) ;;


let rec check_cps_tail_rec_f flist original_k x k e = 
    match e
    with ConstCPS (k', c) -> 
        cont_tail_recursive original_k k' flist
    | VarCPS (k', v) -> 
        (*
        print_string "\nVarCPS\n"; 
        *)
        cont_tail_recursive original_k k' flist
    | MonOpAppCPS (k', mono_op, o1, exk) ->
        (*
        print_string "\nMonOpAppCPS\n"; 
        *)
        cont_tail_recursive original_k k' flist
    | BinOpAppCPS (k', bin_op, o1, o2, exk) -> 
        (*
        print_string "BinOp\n"; 
        *)
        cont_tail_recursive original_k k' flist
    | IfCPS (b, e1, e2) -> 
        (check_cps_tail_rec_f flist original_k x k e1) && (check_cps_tail_rec_f flist original_k x k e2)
    | AppCPS (k', e1, e2, exk) -> 
        (
        (*
        print_string ("AppCPS:\n"^e1^", "^e2^", f: "^f^"\n");
        *)
        if ( List.exists (fun x -> x = e1) flist)
            then 
                (
                if (k'=original_k) 
                    then 
                        (*
                        print_string "true\n";
                        *)
                        true
                     else 
                         (*
                         print_string "false\n";
                         *)
                         false
                )
            else
                cont_tail_recursive original_k k' flist
        )
    | FunCPS (kappa, x, k, ek, e) -> true
    | FixCPS (kappa, f, x, k, ek, e) -> true

and cont_tail_recursive original_k k flist = 
    match k
    with ContVarCPS i -> true
    | External -> true
    | FnContCPS (x, e) -> 
        check_cps_tail_rec_f flist original_k x k e (*TODO x ?*)
    | ExnMatch ek -> true ;;


let check_tail_recursion dec =
    match dec
    with Anon e -> true
    | Let (x,e) -> true
    | LetRec (f,x,e) ->
        let (i,j) = (next_index(),next_index()) 
        in
            let ecps2 = cps_exp e (ContVarCPS i) (ExnContVarCPS j) 
            in
            (*
            print_string ((string_of_exp_cps ecps2)^"\n");
            *)
            let flist = convert_f_exp ecps2 f
                in
                check_cps_tail_rec_f flist (ContVarCPS i) x (ContVarCPS i) ecps2 ;;
                
\end{lstlisting}


\subsection{Code Structure}

The tool \tool has been developed in a series of iterations, and deliberate attempts have been made to keep
the code concise and short.
The total lines of code is about 1800, and the central parts of the code (where we implement the main algorithms), is about 200 lines (100 lines each for direct style and CPS style)

The code has been kept modular, to the best possible extent.
The following is a brief description of the various files:
\begin{itemize}
 \item \texttt{definitions.ml} ; Includes the various definitions for the types used in the tool. 
 This file includes algorithms for unification, type-inference, transforming PicoML expressions in CPS etc., apart from basic utilities for printing various types.
 Most of the parts in this file have been derived from MP7
 \item \texttt{tailRecPicoMLlex.mll} : Includes the code for token generation for PicoML expressions, largely borrowed from ML4
 \item \texttt{tailRecPicoMLparse.mly} : Includes the code for parsing PicoML expressions, largely borrowed from MP5
 \item \texttt{checkTailRec.ml} : Contains the main algorithm for checking PicoML expressions and declarations for tail recursion
 \item \texttt{checkTailRecCPS.ml} : Contains the main algorithm for checking CPS expressions and declarations for tail recursion
 \item \texttt{tailRecPicoMLInt.ml} : Contains utilities for wrapping the main code for interactive use
 \item \texttt{tailRecPicoMLTest.ml} : Contains utilities for wrapping the main code for non-interactive testing
\end{itemize}

\subsection{Comparison with Original Proposal}

Below is the relevant part of the original proposal :

\lstinputlisting[language={}]{proposal.txt}

Clearly, we have improved upon the proposal in the following ways: 

\begin{enumerate}
 \item We have changed a lot of definitions, realizing that the earlier characterizations were unsound, or far from complete, or completely wrong.
 Specific differences have not been pointed out for the sake of conciseness
 \item We have also implemented an algorithm for checking CPS expressions, thus enabling us to check for robustness of the tool.
\end{enumerate}
